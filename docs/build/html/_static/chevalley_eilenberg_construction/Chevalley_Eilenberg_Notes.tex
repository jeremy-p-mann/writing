\documentclass[11pt,reqno]{amsart}
\usepackage{geometry}                % See geometry.pdf to learn the layout options. There are lots.
\geometry{letterpaper}                    
%\usepackage[parfill]{parskip}    % Activate to begin paragraphs with an empty line rather than an indent
\usepackage{graphicx}
\usepackage{amssymb}
\usepackage{epstopdf}
\usepackage{amsmath}   
\usepackage{amsthm}
\usepackage{amsfonts}
\usepackage{latexsym}
\usepackage{color}
\usepackage{blindtext}
\usepackage[toc,page]{appendix}
\usepackage[backref]{enotez}      
\usepackage{hyperref}
\let\footnote=\endnote
\let\footnote=\endnote
\DeclareGraphicsRule{.tif}{png}{.png}{`convert #1 `dirname #1`/`basename #1 .tif`.png}


 
%% SHORTCUTS
\theoremstyle{plain}
\newtheorem{thm}{Theorem}
\newtheorem{prop}[thm]{Proposition}
\newtheorem{lem}[thm]{Lemma}
\numberwithin{equation}{section}
\newtheorem{cor}[thm]{Corollary}

\theoremstyle{remark}
\newtheorem{rek}[thm]{Remark}
\newtheorem{fact}[thm]{Fact}

\theoremstyle{definition}
\newtheorem{defi}[thm]{Definition}
\newtheorem{ex}[thm]{Example}

% FOR DEFINING OPERATORS SUCH AS COLIM
\DeclareMathOperator{\Sym}{Sym}


%% FOR THE SPECTRAL SEQ 
% \usepackage{sseq}

%% BTWN PARAGRAPHS
\setlength{\parindent}{2em}
% \setlength{\parskip}{1em}

% FOR THE COMM DIAG
\usepackage{tikz-cd}

 \tikzset{commutative diagrams/.cd,
mysymbol/.style={start anchor=center,end anchor=center,draw=none}
}
\newcommand\MySymb[2][\bigstar]{%
\arrow[mysymbol]{#2}[description]{#1}}
  
%FOR THE FONTS
\usepackage[mathscr]{euscript}

%FOR THE FOOTNOTES
\makeatletter
\def\@xfootnote[#1]{%
  \protected@xdef\@thefnmark{#1}%
  \@footnotemark\@footnotetext}
\makeatother

%% Make TITLE & Begin Documment
\title{Graph Quantum Mechanics}


%	 MACROS

% Categories

\newcommand{\CAlg}{\mathrm{CAlg}^\mathrm{aug}}
\newcommand{\Spaces}{\mathcal{S}\mathrm{paces}}
\newcommand{\Vect}{\mathrm{Vect}}
\newcommand{\Set}{\mathrm{Set}}
\newcommand{\Mod}{\mathrm{Mod}}
\newcommand{\AlgLie}{\mathrm{Alg}_\mathrm{Lie}}

%Functors
\newcommand{\triv}{\mathrm{triv}}
\newcommand{\trivlie}{\mathrm{triv}_\mathrm{Lie}}
\newcommand{\freelie}{\mathrm{Free}_\mathrm{Lie}}
\newcommand{\Tm}{\mathrm{T}[-1]}
\newcommand{\cT}{\mathrm{T}^\vee}
\newcommand{\T}{\mathrm{T}}
\newcommand{\forms}{\Omega ^\bullet}
\newcommand{\formscl}{\forms_\mathrm{cl}}
\newcommand{\CE}{\mathrm{CE}}
\newcommand{\Cone}{\mathrm{Cone}}

%Sheaves
\newcommand{\C}{\mathrm{C}^\infty}
\newcommand{\muleb}{\mu_{\mathrm{Leb}}}



%Objects
\newcommand{\lieg}{\mathfrak{g}}
\newcommand{\obs}{\mathscr{O}}
\newcommand{\R}{\mathbb{R}}



%Operators
\newcommand{\ddr}{\mathrm{d}_\mathrm{dR}}
\newcommand{\QCE}{\mathrm{Q}}
\newcommand{\MCG}{\Theta_\lieg}







\newcommand{\Unwg}{\tilde{\Gamma}}
\newcommand{\sympg}{\omega_\Gamma}
\newcommand{\Heisg}{\mathrm{Heis} (\Eg, \sympg)}



\newcommand{\mphi}{m_\phi}


\newcommand{\I}{\mathrm{I}}


\newcommand{\E}{\mathscr{E}}
\newcommand{\Egn}{\Eg^{\neq 0 }}
\newcommand{\Eg}{\mathscr{E}_\Gamma}
\newcommand{\Egv}{\mathscr{E}_\Gamma^\vee}
\newcommand{\Egj}{\mathscr{E}_{\Gamma, \J}}
\newcommand{\Rj}{\R \{j_0\}}
\newcommand{\Rgz}{\R^{\Gamma_0}}
\newcommand{\X}{\mathscr{X}}
\newcommand{\Ga}{\mathbb{G}_a}

\newcommand{\partf}{\mathrm{Z}}

\newcommand{\cofib}{\mathrm{Cofib}}
\newcommand{\maps}{\mathrm{Maps}}


\newcommand{\B}{\mathrm{B}}
\newcommand{\Spec}{\mathrm{Spec}}
\newcommand{\Ukmod}{\mathscr{U}_\mathrm{Mod_k}}
\newcommand{\Tst}{\mathrm{T}^*}
\newcommand{\fun}{\mathscr{O}}

\newcommand{\mx}{\mathfrak{m}_X}



\title{Chevalley-Eilenberg Notes}

\begin{document}
\maketitle

The basic goal of this theory will be 

\section{Preliminary Whispers}



The standard narrative for the Chevalley-Eilenberg construction consists of constructing a finite dimensional model of the cohomology of a compact connected Lie group, $G$ in terms of it's Lie algebra, $\lieg$. 
%
\footnote{For our purposes, $\lieg := \mathrm{T}_eG$. The relation between $\lieg$ in this sense and left invariant vector fields will follow shortly.}
%

\subsection{Cohomology of Compact Lie Groups}
We begin with the following technically useful lemma:

\begin{lem}\label{bivmet}
$G$ has a bi-invariant metric and volume form. 
\end{lem}

With this metric, parallel transport gives a map:
$$
\lieg^\vee \hookrightarrow \Omega ^1(G)
$$
Along with:
$$
\lieg \hookrightarrow \X(G)
$$
It's not difficult to show that the image of this map is closed under the Lie bracket, so that $\lieg$ obtains the structure of a Lie algebra. Our next goal will be to 

\begin{defi}
Let the \textit{Maurer-Cartan form}, $\Theta_G \in \Omega^1(G) \otimes \lieg$ denote the adjoint of the first equation above. 
\end{defi}

\begin{rek}
For now on, everything we say must be taken in a graded commutative sense. For example, we can view the map encoded by the Maurer-Cartan form as
$$
\lieg^\vee[-1] \rightarrow \Omega^1 (G) 
$$
and the Maurer-Cartan form as living inside:
$$
\MCG \in \Omega^1 (G) \otimes \lieg [1]
$$
\end{rek}

By definition, the above map uniquely extends to a map of graded commutative algebras:
$$
\Sym(\lieg^\vee [-1]) \rightarrow \forms(G)
$$
\begin{rek}
One can use formal nonsense to construct an equivalence:
$$
\Sym^k(\lieg^\vee[-1])\simeq \Big( \wedge^k (\lieg[-1]) \Bigl)^\vee
$$
Note that this isomorphism requires dividing by the symmetric group.
\end{rek}

For obvious reasons, we'd like to extend this to a map of commutative differential graded Lie algebras. In order for this to happen, we need only fill in the left map of the commutative diagram:
$$
\begin{tikzcd}
\Sym^2(\lieg^\vee[-1]) \arrow[r]    & \Omega^2(G)                   \\
\lieg^\vee[-1] \arrow[r, "\MCG"] \arrow[u, "\QCE"] & \Omega^1(G) \arrow[u, "\ddr"]
\end{tikzcd}
$$
which determines it's behavior on all of $\Sym(\lieg^\vee[-1])$. This is relatively, straightforward, using the explicit formula for the deRham differential in terms of the bracket of vector fields. The result is that:
$$
\QCE = [-,-]^\vee
$$

In other words, the Maurer -Cartan form encodes the lie bracket. In the notation of differential operators:
$$
\QCE
%
\footnote{A strictly necessary exercise is to show the relationship between the equation $\ddr^2 = 0$ and the Jacobi-identity.} 
%
 = c^k_{i, j} c^k \frac{\partial^2}{\partial c_i \partial c_j}
$$
where $c^k$ is a basis for $\lieg^\vee$, and $c^k_{i, j}$ are the structure constants for $\lieg$. Therefore, we can recover the Lie structure via Taylor coefficients:
$$
 \frac{\partial^2}{\partial c_i \partial c_j}|_{c=0}\QCE(c^k)= c^k_{i, j} 
$$



Obviously we can view $\MCG \in \forms(G) \otimes \lieg$, which is a differential graded Lie algebra with differential $\ddr$. The commutativity of the above square may be expressed as the infamous
\begin{lem}{\textbf{Maurer-Cartan Equation}}
$$
\ddr \MCG = - \frac{1}{2}[\MCG, \MCG]
$$
expressing the fact that it's self-commutator is null-homotopic in $\forms(G) \otimes \lieg$
\end{lem}

\begin{defi}
Given a lie algebra $\lieg$, the \textit{Chevalley-Eilenberg cochains} on $\lieg$ is the commutative algebra:
%
\footnote{for now on, everything we say will be commutative with respect to the standard . In other words, Koszul sign rules will be implicit in our discussion}
%
$$
\CE^*(\lieg) := \Bigl( \Sym (\lieg[-1]), \QCE)
$$
\end{defi}

The construction of an explicit contraction
%
\footnote{The author is not going through this because they're lazy. It's a really gorgeous contraction, an absolute must-see blockbuster of a technique.}
%
 (which requires that $G$ be compact) shows that

\begin{thm}
The inclusion, determined by the Maurer-Cartan form, induces an equivalence:
$$
\CE^*(\lieg) \simeq \forms(G)
$$
\end{thm}
which is pretty cool, as the right hand side is infinite dimensional, while the left hand side is finite dimensional. It further shows that we have an algebra description for the rational homotopy type of $\Omega \mathrm{B}G$. 

Finally, we can see that the multiplication comes from the diagonal map:
$$
\CE^*(\lieg \rightarrow \lieg \times \lieg)
$$

\subsection{Deformations}

Note that every $\eta \in \CE^2(\lieg)$ gives a map:
$$
\wedge^2 \lieg \overset{\eta}\longrightarrow \R 
$$
We can use this to ``deform" $\lieg$. In other words, construct a Lie algebra structure on:
$$
\lieg \oplus \R\cdot\hbar
$$
So that:
$$
[c_0, c_1]_\eta = [c_0, c_1] + \eta (c_0c_1)\hbar
$$
and $1$ commutes with everything else. 

In other words, we have a fibre sequence of Lie algebras:
$$
\R\cdot \hbar \rightarrow \lieg^\eta \rightarrow \lieg
$$
which is conventionally referred to as a central extension. 

Of course, not all antisymmetric pairings give Lie algebra extensions extensions. An explicit computation shows that this occurs when:
$$
\QCE\eta = 0 
$$
Moreover, we can use the data (if it exists) $\QCE \eta_1 = \eta$ to construct an equivalent of Lie algebras:
$$
\lieg^\eta \simeq \lieg \oplus \R \cdot \hbar
$$

The most famous example of such a central extension comes from the symplectic form of a symplectic vector space. Here, the $g$ are linear functions on the symplect form vanishing at the origin, and the Poisson bracket gives the central extension. Lie algebras of this type are commonly referred to as Heisenberg Lie algebras, although this conceptualization is due to Dirac.
%
\footnote{Poisson brackets, despite being very helpful computational tools in classical dynamics, was esoteric and almost forgotten in Dirac's time. He had to dig into the library to find a reference on it. }
% 


\section{The General Case}

We go through a more general description of the above construction.

\begin{rek}
Let's fix a finite group $G$. We'll denote the category of G-Sets as $\Set_G$. We can view every set as a $G-\Set$, via the trivial action, giving a functor:
$$
\Set \overset{\triv_G}\rightarrow \Set_G
$$
One can check that this functor admits a left adjoint:
$$
X \rightarrow X/G
$$
Therefore, we can think of the quotient as being \textit{defined} as the left adjoint to the trivial action functor. 
\end{rek}

We can repeat the above story, but for an augmented associative algebra $A \rightarrow k$, making $k$ into an $A$-module. In this case, we have a map:
$$
\Mod_k \overset{\mathrm{res}_{A\rightarrow k}}\longrightarrow \Mod_A 
$$
Which admits a left adjoint:
\begin{align*}
\Mod_A &\rightarrow \Mod_k \\
M &\rightarrow k \otimes_A M
%
\footnote{The reader will excuse us for omitting the derived language. We of course mean derived tensor product. This can be abstractly expressed as a bar construction, but the convention approach is to just }
%
\end{align*}

The universal enveloping algebra, $U(\lieg\rightarrow 0)$, provides an interesting example of augmented associated algebras. Moreover, it's not hard to convince oneself that $\lieg$-representations are equivalent to modules over the universal enveloping algebra. Therefore, we obtain a functor:
$$
\mathrm{Rep}_\lieg \rightarrow \Mod_{U(\lieg)} \overset{k \otimes_{U(\lieg)}(-)} \longrightarrow \Mod_k
$$

\section{An Explicit Model}

Although the above definitions has a few advantages, in applications, one wants to work within an explicit model.
%
\footnote{Here, explicit model means a graded vector space with a differential and an algebraic structure which exists on-the-nose.}
%
In lieu of going through the explicit construction, we outline the essential ingredients, and simply give the explicit model. 
%
\footnote{Many authors take the pedagogically responsible route of simply stating this model as the definition.}
%

The basic idea is to construct a cofibrant of the trivial module given by the universal enveloping algebras augmentation $U(\lieg)\rightarrow k $. This is accomplished by constructing a contractible Lie algebra equipped with an inclusion from $\lieg$, and apply the universal enveloping algebra functor. The underlying graded chain complex is simply the cofiber of $\lieg \rightarrow 0$. 
%
\footnote{Whose underlying graded vector space is $\lieg \oplus \lieg[1]$. }
%
The trick is to construct a Lie structure on this cofiber. This is conventonally refered to as $\Cone(\lieg)$ for obvious reasons. With this in place, after a check of model categorical stuff, we see that 
$$
k \otimes_{U(\lieg)} M \simeq U(\Cone(\lieg)) \otimes^\mathrm{ps}_{U(\lieg)} M
$$
From this, some ore homological manipulations produce the following description.


As this discussion was meant to be explicit, we'll give our description in terms of a basis for $\lieg$ and $M$ and structure constants. For simplicity, we'll restrict ourselves to lie algebras and representations concentrated in degree 0. 

\begin{itemize}
\item $\lieg = k\cdot \{c_i\}$
\item $M = k\cdot \{m_n\}$
\item $[c_i, c_j] = c^k_{ij}c_k$
\item $c_i \cdot m_n = m^k_{in} m_k$
\end{itemize}

\begin{defi}
$$
\CE_*(\lieg, M) := \Bigl( \Sym(\lieg[1] )\otimes M, c^k_{ij}\frac{\partial^2}{\partial c_i \partial c_j}+m^k_{in} m_k \frac{\partial ^2}{\partial c_i \partial  m_n}\Big)
$$
We refer to this differential as the \textit{Chevalley-Eilenberg differential}, and this chain complex as \textit{Chevalley-Eilenberg chains with coefficients in $M$} or \textit{Lie algebra chains with coefficients in $M$}. When $M$ is the trivial 1-dimensional representation will be denoted:
$$
\CE_*(\lieg)
$$
The homology groups of these linear objects are referred to as \textit{Lie homology}
\end{defi}

Using this model, we see that:
\begin{lem}
$$
H_0 \Big(\CE_*(\lieg, M) \Bigl) = M/[\lieg, M] 
$$
\end{lem}
The right-hand side is usually referred to as the annihilator, and is a relative coarse invariant of the representation.

We can immediately read off a few properties of this functor:
\begin{lem} \label{CEprops}
There are equivalences:
\begin{itemize}
\item $\CE_*(\freelie(V)) \simeq k \oplus V[1]$
\item $\CE_* (\trivlie (V))
%
\footnote{Trivial means abelian. It has "trivial brackets"}
% 
\simeq \Sym (V[1])$
\item $\CE_* ( \lieg \times \lieg') \simeq \CE_* (\lieg) \otimes \CE_*(\lieg')
%
\footnote{This equivalence requires reordering and splitting monomials into pieces coming from $\lieg$ and $\lieg'$. Therefore it involves some combinatorics of partitioning and Koszul sign rules}
%
$
\end{itemize}
\end{lem}

Some nontrivial category theory establishes the following:
\begin{thm}
The Chevalley-Eilenberg construction:
$$
\CE_*(-): \AlgLie \rightarrow \Mod_k
$$
preserves colimits. 
\end{thm}

The third property of \ref{CEprops} implies that we can endow $\CE_*(\lieg)$ with the structure of a co-(unital commutative algebra) via:
$$
\CE_*(0 \rightarrow \lieg \rightarrow \lieg \times \lieg) 
$$
For example:
$$
\Delta(c_i) = 1 \otimes c_i + c_i \otimes 1
$$
\begin{lem}
The Chevalley-Eilenberg chains on a Lie algebra has a canonical co-(augmented commutative algebras) structure. In other words, it factors through co-(augmented commutative algebras). 
\end{lem}

\begin{rek}
In particular, Chevalley-Eilenberg chains on the trivial lie algebra is the co-(free augmented commutative algebra). 
\end{rek}



\newpage

\printendnotes

\end{document}